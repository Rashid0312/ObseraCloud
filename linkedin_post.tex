\documentclass{article}
\usepackage[utf8]{inputenc}
\usepackage{hyperref}
\usepackage{enumitem}

\title{ObseraCloud: A Distributed Observability Platform}
\author{Abdirashiid Sammantar}
\date{}

\begin{document}

\maketitle

\noindent ObseraCloud is a \textbf{proof-of-concept (POC)} distributed observability platform designed to unify logs, metrics, and traces into a single correlated view. It demonstrates how high-performance ingestion and actionable visibility can be architected from scratch using modern open-source tools.

\section*{Core Architecture \& Tech Stack}

The system is built on a polyglot microservices architecture optimized for specific workload requirements:

\begin{itemize}
    \item \textbf{Ingestion Layer (Go):} Leveraging Go's concurrency model to handle high-throughput log and trace streams with minimal latency.
    \item \textbf{Backend Logic (Python):} Utilizing Python for complex data processing and normalization tasks.
    \item \textbf{Storage (ClickHouse \& PostgreSQL):} Adopting ClickHouse for columnar storage to enable sub-second queries on massive log datasets, backed by PostgreSQL for consistent relational data.
    \item \textbf{Frontend (React \& TypeScript):} A responsive, real-time dashboard that streams data updates via WebSockets.
\end{itemize}

\section*{Key Features}

\begin{itemize}
    \item \textbf{Distributed Tracing:} Implements an OpenTelemetry-compatible gateway to correlate requests across microservices.
    \item \textbf{Unified Data:} Eliminates data silos by linking logs directly to their corresponding traces.
    \item \textbf{Scalability:} Designed to handle backpressure and data spikes inherent in cloud environments.
\end{itemize}

\noindent ObseraCloud demonstrates a practical approach to building scalable, enterprise-grade observability tools using open-source technologies.

\vspace{1em}
\noindent Repository and documentation: \href{[Insert GitHub Link]}{[Insert GitHub Link]}

\vspace{1em}
\noindent \#SoftwareEngineering \#SystemArchitecture \#Golang \#Python \#React \#OpenTelemetry \#DevOps \#ObseraCloud

\end{document}
